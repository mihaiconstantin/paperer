% Decrease font size.
\small

% Section title.
\section{Outline}

% Legend.
\subsection{Symbol Legend}

% Symbol meaning.
\begin{paragraphs}
    \item Not started or in rough form (i.e., to write).
    \item[\done] Ready for review or in final form.
    \item[\drop] Will not be written or to be dropped.
    \item[\look] Needs attention or input (i.e., something is not clear).
    \item[\head] Grouping heading (e.g., section, subsection etc.).
\end{paragraphs}

% Structure.
\subsection{Paper Structure}

% Outline items.
\begin{paragraphs}
    % Introduction section.
    \item[\head] \textbf{Introduction}
    \begin{paragraphs}
        \item[\done] \pr{1} the topic sentence for paragraph one
        \item[\done] \pr{2} the topic sentence for paragraph two
    \end{paragraphs}

    % Background section.
    \item[\head] \textbf{Background}
    \begin{paragraphs}
        \item[\look] \pr{3} the topic sentence for paragraph three

        % Subsection one.
        \item[\head] \textbf{Subsection One}
        \begin{paragraphs}
            \item[\done] \pr{4} the topic sentence for paragraph four
            \item \pr{5} the topic sentence for paragraph five
            \begin{notes}
                \item Some notes about what should be included in this
                    paragraph. Their visibility can be toggled via the
                    \code{shownotes} boolean as \code{boolfalse\{shownotes\}}.
                    \lipsum[7][1-5]
            \end{notes}
        \end{paragraphs}

        % Subsection two.
        \item[\head] \textbf{Subsection Two}
        \begin{paragraphs}
            \item[\done] \pr{6} the topic sentence for paragraph six
        \end{paragraphs}
    \end{paragraphs}

    % Another section.
    \item[\head] \textbf{\dots}
    \begin{paragraphs}
        \item ...
    \end{paragraphs}
\end{paragraphs}

% Reset font size.
\normalsize
